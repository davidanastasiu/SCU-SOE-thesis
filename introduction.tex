\chapter{Introduction}

\section{Motivation}

This document is a LaTeX template created by David C. Anastasiu for thesis and project report authors from the College of Science and Engineering (CSEN) at Santa Clara University (SCU). The template extends an initial LaTeX class created by Prof. Darren Atkinson and attempts to simplify the formatting of theses for students while also guiding them in the writing process. Students should, however, read the current library thesis style requirements carefully and rectify any style discrepancies before submitting their theses. Additionally, note that reviewers are very strict on grammar and spelling mistakes.


\section{Solution}

The present ``thesis'' serves as a visual example of what the finished thesis product should look like. In addition, it provides some guidance on how to use the scu-thesis style, the provided LaTeX template project, and LaTeX in general to produce a quality thesis that complies with submission requirements. LaTeX, in combination with the provided style, automate many of the style aspects of the thesis, as long as students do not modify the style or introduce new style elements manually in the text. For example, one can make all section titles bold by simply putting the text within a \texttt{{\textbackslash}textbf} block. However, that would break the style and cause the thesis to be returned for edits by the reviewers. Students are thus, once again, urged to refrain from style modifications and to double-check style compliance against the style guide before submitting their theses.

Throughout the thesis, we will make use of tables and figures from some of my own papers and my students' theses. These tables and figures will likely be taken out of context and should not be considered relevant to the topic of this thesis. They are simply provided as examples for how to properly include these elements in a thesis document.

The remainder of this thesis is structured as follows. First, in Chapter~\ref{ch:literature_review}, we will introduce the reader to the task of reviewing literature that is relevant to the chosen topic of study, with the goal of identifying the current state-of-the-art method or system that potentially solves the problem at hand. This is the basis of all research projects, as we should not ``re-invent the wheel'' and try not to repeat mistakes made by others. Chapter~\ref{ch:methods} serves as a brief introduction to writing documents in the TeX macro language. Additional example chapters are included for reference and should be edited/changed as necessary for your thesis/project. Finally, Chapter~\ref{ch:future_work} describes to-do items envisioned for this project and Chapter~\ref{ch:conclusion} concludes this thesis.